\documentclass[12pt,onecolumn]{IEEEtran}
\usepackage{ifpdf}
\usepackage{graphicx}
\usepackage{cite}
\usepackage{listings}
\usepackage{color}
\usepackage{subcaption}
\usepackage{amsmath}
\usepackage{epstopdf}
\usepackage{algorithm2e}

\definecolor{mygreen}{rgb}{0,0.6,0}
\definecolor{mygray}{rgb}{0.5,0.5,0.5}
\definecolor{mymauve}{rgb}{0.58,0,0.82}

\lstset{ 
	backgroundcolor=\color{white},   % choose the background color; you must add \usepackage{color} or \usepackage{xcolor}; should come as last argument
	basicstyle=\footnotesize,        % the size of the fonts that are used for the code
	breakatwhitespace=true,          % sets if automatic breaks should only happen at whitespace
	breaklines=true,                 % sets automatic line breaking
	captionpos=b,                    % sets the caption-position to bottom
	commentstyle=\color{mygreen},    % comment style
	deletekeywords={...},            % if you want to delete keywords from the given language
	escapeinside={\%*}{*)},          % if you want to add LaTeX within your code
	extendedchars=true,              % lets you use non-ASCII characters; for 8-bits encodings only, does not work with UTF-8
	firstnumber=1,                   % start line enumeration with line 1
	frame=single,                    % adds a frame around the code
	keepspaces=true,                 % keeps spaces in text, useful for keeping indentation of code (possibly needs columns=flexible)
	keywordstyle=\color{blue},       % keyword style
	language=Java,                   % the language of the code
	morekeywords={*,...},            % if you want to add more keywords to the set
	numbers=left,                    % where to put the line-numbers; possible values are (none, left, right)
	numbersep=5pt,                   % how far the line-numbers are from the code
	numberstyle=\tiny\color{mygray}, % the style that is used for the line-numbers
	rulecolor=\color{black},         % if not set, the frame-color may be changed on line-breaks within not-black text (e.g. comments (green here))
	showspaces=false,                % show spaces everywhere adding particular underscores; it overrides 'showstringspaces'
	showstringspaces=false,          % underline spaces within strings only
	showtabs=false,                  % show tabs within strings adding particular underscores
	stepnumber=5,                    % the step between two line-numbers. If it's 1, each line will be numbered
	stringstyle=\color{mymauve},     % string literal style
	tabsize=1,	                   % sets default tabsize to 2 spaces
	title=\lstname                   % show the filename of files included with \lstinputlisting; also try caption instead of title
}

\ifpdf
\pdfinfo{
	/Title (The Modelling of a Linear DC Motor to be Used in Space)
	/Author (Nicholas Kastanos (1393410))
	/CreationDate (D:201902211949)
	/ModDate (D:201902211949)
	/Subject (Software 2 Report, Modelling)
	/Keywords (modelling, linear DC motor, control)
}
\fi

\begin{document}
	\title{Laboratory Work 3: MapReduce with Phoenix++}
	
	\author{\IEEEauthorblockN{Anita de Mello Koch (1371116)\\Nicholas Kastanos (1393410)\\Brendon Swanepoel (601949)}\\
		\IEEEauthorblockA{School of Electrical \& Information Engineering, University of the Witwatersrand, Private Bag 3, 2050, Johannesburg, South Africa\\ELEN4020 Data Intensive Computing in Data Science\\15 April 2019}}
	
	\maketitle
	\thispagestyle{empty}
	
	\section{Introduction}
	

	
	\section{}
	
	
	
	\section{Conclusion}
	
	
	
	\clearpage
	\appendices
	\renewcommand\thefigure{\thesection.\arabic{figure}} 
	\renewcommand\theequation{\thesection.\arabic{equation}} 
	\renewcommand\thetable{\thesection.\arabic{table}}
	\renewcommand\thelstlisting{\thesection.\arabic{lstlisting}}
	
	\section{Pseudocode for Word Count}
	\setcounter{figure}{0}  
	\setcounter{equation}{0} 
	\setcounter{table}{0}

	\begin{algorithm}[H]
		\SetAlgoLined
		\SetKwData{Left}{left}\SetKwData{This}{this}\SetKwData{Up}{up}
		\SetKwFunction{Union}{Union}\SetKwFunction{FindCompress}{FindCompress}
		\SetKwInOut{Input}{input}\SetKwInOut{Output}{output}
		\Input{Section of text in each bin.}
		\Output{Text separated into single words.}
		\ForEach{character in input}{Change character to lower-case;\\}
		\While{iterator is not at end of input}{
			\While{iterator is not at end of input \& character is not aphabetic}{
				Increase iterator;\\
			}
			Set beginning of word to iterator;\\
			\While{interator is not at end of input \& (character is alphabetic or apostrophe)}{
				Increase iterator;\\
			}
			\If{iterator is not beginning of word}{Output word;}
		}
	\caption{Map function of the word count program.}
	\end{algorithm}
\vspace{5mm}
	\begin{algorithm}[H]
		\SetAlgoLined
		\SetKwData{Left}{left}\SetKwData{This}{this}\SetKwData{Up}{up}
		\SetKwFunction{Union}{Union}\SetKwFunction{FindCompress}{FindCompress}
		\SetKwInOut{Input}{input}\SetKwInOut{Output}{output}
		\Input{All input text data.}
		\Output{Text data separated into bins.}
		\If{there is no more data to separate}{exit;\\}
		\While{iterator is less than the chunk size \& not the end of the file}{increase iterator;\\}
		Output chunk to bin;\\
		\vspace{2mm}
		\caption{Split function for the word count program.}	
	\end{algorithm}
\vspace{5mm}
	
\end{document}